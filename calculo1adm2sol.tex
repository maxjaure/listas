\documentclass[12pt,a4paper]{article}
\usepackage[portuguese]{babel}
\usepackage[margin=1in]{geometry}
\usepackage[utf8]{inputenc}
\usepackage{amsmath}
\usepackage{amsthm}
\usepackage{amssymb}
\usepackage{amsfonts}
\usepackage[colorlinks,allcolors=blue]{hyperref}
\usepackage{microtype}
\usepackage{graphicx}

%\addtokomafont{disposition}{\rmfamily}
\let\emptyset=\varnothing

\newcommand{\tb}{\textbf}
\newcommand{\tbu}[1]{\tb{\textup{#1}}}

\newcommand{\dpar}[1]{\left(#1\right)}
\newcommand{\dsqr}[1]{\left[#1\right]}
\newcommand{\dcur}[1]{\left\{#1\right\}}
\newcommand{\dabs}[1]{\left|#1\right|}
\newcommand{\ang}[1]{\left\langle#1\right\rangle}

\newcommand{\ds}{\displaystyle}

\newcommand{\N}{\mathbb{N}}
\newcommand{\Z}{\mathbb{Z}}
\newcommand{\Q}{\mathbb{Q}}
\newcommand{\R}{\mathbb{R}}

\DeclareMathOperator{\sen}{sen}
\DeclareMathOperator{\arcsen}{arc sen}
\DeclareMathOperator{\senh}{senh}

\title{Respostas da segunda lista de matemática II}
\author{Prof.: Max Jáuregui}
\date{}

%\linespread{1.2}
%\linespread{1.213} %11pt
%\linespread{1.241} %12pt

\begin{document}
\maketitle

\begin{enumerate}
  \item $(f\circ g)(x)=(2x+3)^2-2(2x+3)$, $(g\circ f)=2(x^2-2x)+3$.
  \item[3.]
  \begin{enumerate}
    \item $f^{-1}:\R\to\R$, $f^{-1}(y)=(y-7)/2$
    \item $g^{-1}:[0,\infty)\to [-1,\infty)$, $g^{-1}(y)=y^2-1$
    \item $h^{-1}:(-\infty,2]\to [0,\infty)$, $h^{-1}(y)=\sqrt{2-y}$
  \end{enumerate}
  \item[4.] $\R$ para todas.
  \item[6.] Determine a imagem das seguintes funções quadráticas:
  \begin{enumerate}
    \item $[3/4,\infty)$
    \item $[5/3,\infty)$
    \item $(-\infty,1]$
    \item $(-\infty,11/4]$
  \end{enumerate}
  \item[7.] Determine o domínio das seguintes funções:
  \begin{enumerate}
    \item $\R\setminus \{7/3\}$
    \item $\R\setminus \{1/2,-5\}$
    \item $\R\setminus\{1,3\}$
    \item $[-8/5,\infty)$
    \item $[-1,2/3]$
    \item $(-\infty,-3]\cup[1,\infty)$
    \item $\ds\dsqr{\frac{3-\sqrt{17}}{2},\frac{3+\sqrt{17}}{2}}$
  \end{enumerate}
\end{enumerate}
\end{document}

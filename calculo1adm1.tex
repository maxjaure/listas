\documentclass[12pt,a4paper]{article}
\usepackage[portuguese]{babel}
\usepackage[margin=1in]{geometry}
\usepackage[utf8]{inputenc}
\usepackage{amsmath}
\usepackage{amsthm}
\usepackage{amssymb}
\usepackage{amsfonts}
\usepackage[colorlinks,allcolors=blue]{hyperref}
\usepackage{microtype}
\usepackage{graphicx}

%\addtokomafont{disposition}{\rmfamily}
\let\emptyset=\varnothing

\newcommand{\tb}{\textbf}
\newcommand{\tbu}[1]{\tb{\textup{#1}}}

\newcommand{\dpar}[1]{\left(#1\right)}
\newcommand{\dsqr}[1]{\left[#1\right]}
\newcommand{\dcur}[1]{\left\{#1\right\}}
\newcommand{\dabs}[1]{\left|#1\right|}
\newcommand{\ang}[1]{\left\langle#1\right\rangle}

\newcommand{\ds}{\displaystyle}

\newcommand{\N}{\mathbb{N}}
\newcommand{\Z}{\mathbb{Z}}
\newcommand{\Q}{\mathbb{Q}}
\newcommand{\R}{\mathbb{R}}

\DeclareMathOperator{\sen}{sen}
\DeclareMathOperator{\arcsen}{arc sen}
\DeclareMathOperator{\senh}{senh}

\title{Primeira lista de matemática II}
\author{Prof.: Max Jáuregui}
\date{}

%\linespread{1.2}
%\linespread{1.213} %11pt
%\linespread{1.241} %12pt

\begin{document}
\maketitle

\begin{enumerate}
  \item Encontre os intervalos formados pelos valores de $x$ que satisfazem cada uma das seguintes desigualdades:
  \begin{enumerate}
    \item $5-2x>3$
    \item $3x+6<5$
    \item $7x-1\ge 2$
    \item $-5x+3\le 2$
  \end{enumerate}
  \item Encontre os intervalos formados pelos valores de $x$ que satisfazem os seguintes sistemas de desigualdades:
  \begin{enumerate}
    \item $3-2x<5$ e $4+2x>6$
    \item $3-2x<5$ e $4+2x\le 6$
    \item $3-2x\le 5$ e $3x+11<5$
    \item $5-3x\le 2$ e $7x-4\le 3$
  \end{enumerate}
  \item Encontre os conjuntos dos valores de $x$ que satisfazem cada uma das seguintes desigualdades:
  \begin{enumerate}
    \item $(x-1)(x+4)\le 0$
    \item $3(x+1)(2x-3)>0$
    \item $10x(2x+1)<0$
    \item $x+|x-1|>1$
    \item $|x|+|x-1|>1$
    \item $|x+1|-|x-1|\le 2$
  \end{enumerate}
  \item Dados os conjuntos $X=\{1,3,5,7\}$ e $Y=\{-1,0,1\}$, construa uma função $f:X\to Y$.
  \item Dado os conjuntos $X=\{2,4,6,8\}$ e $Y=\{n\in\N:n\le 10\}$, construa uma função $f:X\to Y$ cuja imagem seja o conjunto $\{2,5,7\}$.
  \item Na questão anterior é possível definir uma função $g:X\to Y$ cuja imagem seja o próprio conjunto $Y$?
  \item Uma elipse pode ser o gráfico de uma função? Justifique.
  \item Dado o conjunto $X=\{-2,-1,0,1,2\}$, construa o gráfico da função $f:X\to\R$ definida por $f(x)=x^2$.
\end{enumerate}

\end{document}

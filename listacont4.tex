\documentclass[10pt,a4paper]{article}
\usepackage[portuguese]{babel}
%\usepackage[margin=1in]{geometry}
\usepackage[utf8]{inputenc}
\usepackage{amsmath}
\usepackage{amsthm}
\usepackage{amssymb}
\usepackage{amsfonts}
\usepackage[colorlinks,allcolors=blue]{hyperref}
\usepackage{microtype}
\usepackage{graphicx}

%\addtokomafont{disposition}{\rmfamily}
\let\emptyset=\varnothing

\newcommand{\tb}{\textbf}
\newcommand{\tbu}[1]{\tb{\textup{#1}}}

\newcommand{\dpar}[1]{\left(#1\right)}
\newcommand{\dsqr}[1]{\left[#1\right]}
\newcommand{\dcur}[1]{\left\{#1\right\}}
\newcommand{\dabs}[1]{\left|#1\right|}
\newcommand{\ang}[1]{\left\langle#1\right\rangle}

\newcommand{\ds}{\displaystyle}

\newcommand{\N}{\mathbb{N}}
\newcommand{\Z}{\mathbb{Z}}
\newcommand{\Q}{\mathbb{Q}}
\newcommand{\R}{\mathbb{R}}

\DeclareMathOperator{\sen}{sen}
\DeclareMathOperator{\arcsen}{arc sen}
\DeclareMathOperator{\senh}{senh}

\title{Quarta lista de matemática}
\author{Prof.: Max Jáuregui}
\date{}

%\linespread{1.2}
%\linespread{1.213} %11pt
%\linespread{1.241} %12pt

\begin{document}
\maketitle
\begin{enumerate}
  \item Desenhe as circunferências dadas pelas seguintes equações, determinando o raio e a posição do centro em cada caso:
  \begin{enumerate}
  	\item $x^2+y^2=9$
  	\item $(x-2)^2+(y-3)^2+16$
  	\item $(x+2)^2+(y-1)^2=6$
  \end{enumerate}
\item Desenhe as elipses dadas pelas seguintes equações, determinando em cada caso a posição do centro e os comprimentos dos semieixos maior e menor:
\begin{enumerate}
	\item $\ds \frac{x^2}{4}+\frac{y^2}{9}=1$
	\item $\ds \frac{(x-1)^2}{9}+y^2=1$
	\item $4(x+1)^2+(y-1)^2=16$
\end{enumerate}
\item Desenhe as parábolas dadas pelas seguintes equações, determinando em cada caso a distância $f$ da origem ao foco:
\begin{enumerate}
	\item $\ds y=\frac{x^2}{16}$
	\item $\ds x=\frac{y^2}{12}$
	\item $\ds 4y+x^2=0$
	\item $\ds y-2x^2=0$
\end{enumerate}
\item Em cada um dos seguintes casos identifique a cônica que a equação representa (circunferência, elipse, parábola ou hipérbole). No caso de se ter elipse, parábola ou hipérbole, faça um esboço do desenho (sem se preocupar com os eixos de coordenadas) para visualizar a sua orientação (vertical, horizontal, para à direita ou para esquerda).
\begin{enumerate}
	\item $x^2-y^2-9=0$
	\item $4y+2x^2=0$
	\item $2y^2-(x+2)^2=8$
	\item $x^2+(y-1)^2-9=0$
	\item $3x+6y^2=0$
	\item $2(x+1)^2+4y^2-16=0$
	\item $\ds \frac{x^2}{2}+\frac{(y+2)^2}{8}=2$
\end{enumerate}
\end{enumerate}
\end{document}

\documentclass[10pt,a4paper]{article}
\usepackage[portuguese]{babel}
%\usepackage[margin=1in]{geometry}
\usepackage[utf8]{inputenc}
\usepackage{amsmath}
\usepackage{amsthm}
\usepackage{amssymb}
\usepackage{amsfonts}
\usepackage[colorlinks,allcolors=blue]{hyperref}
\usepackage{microtype}
\usepackage{graphicx}

%\addtokomafont{disposition}{\rmfamily}
\let\emptyset=\varnothing

\newcommand{\tb}{\textbf}
\newcommand{\tbu}[1]{\tb{\textup{#1}}}

\newcommand{\dpar}[1]{\left(#1\right)}
\newcommand{\dsqr}[1]{\left[#1\right]}
\newcommand{\dcur}[1]{\left\{#1\right\}}
\newcommand{\dabs}[1]{\left|#1\right|}
\newcommand{\ang}[1]{\left\langle#1\right\rangle}

\newcommand{\ds}{\displaystyle}

\newcommand{\N}{\mathbb{N}}
\newcommand{\Z}{\mathbb{Z}}
\newcommand{\Q}{\mathbb{Q}}
\newcommand{\R}{\mathbb{R}}

\DeclareMathOperator{\sen}{sen}
\DeclareMathOperator{\arcsen}{arc sen}
\DeclareMathOperator{\senh}{senh}

\title{Terceira lista de matemática}
\author{Prof.: Max Jáuregui}
\date{}

%\linespread{1.2}
%\linespread{1.213} %11pt
%\linespread{1.241} %12pt

\begin{document}
\maketitle
\begin{enumerate}
  \item Desenhe as retas dadas pelas seguintes equações:
  \begin{enumerate}
  	\item $y=2x+4$
  	\item $y=-3x+1$
  	\item $\ds y=\frac{x}{2}-2$
  	\item $y=-2x-1$
  \end{enumerate}
\item Encontre a equação das retas com as seguintes informações:
\begin{enumerate}
	\item A reta tem inclinação $5$ e intercepto $3$.
	\item A reta tem inclinação $-2$ e passa pelo ponto $(1,2)$.
	\item A reta tem intercepto $-1$ e passa pelo ponto $(2,3)$.
	\item A reta passa pelos pontos $(0,1)$ e $(1,3)$.
	\item A reta passa pelos pontos $(-1,2)$ e $(3,-1)$. 
\end{enumerate}
\item Dada uma reta com equação $y=2x+3$, encontre a equação de uma reta paralela que
\begin{enumerate}
	\item passe pelo ponto $(1,3)$;
	\item tenha intercepto $-2$.
\end{enumerate}
\item Dada uma reta com equação $y=-2x+2$, encontre a equação de uma reta perpendicular que
\begin{enumerate}
	\item passe pelo ponto $(0,3)$;
	\item tenha intercepto $1$.
\end{enumerate}
\item Encontre o ponto de interseção das retas dadas pelas seguintes equações:
\begin{enumerate}
	\item $y=3x+1$ e $y=-2x+1$;
	\item $y=-2x+3$ e $y=-x-1$;
	\item $y=x+5$ e $y=-2x+2$.
	\item $y=2x-3$ e $\ds y=-\frac{x}{2}+1$
\end{enumerate}
\end{enumerate}
\end{document}

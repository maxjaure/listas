\documentclass[12pt,a4paper]{article}
\usepackage[portuguese]{babel}
\usepackage[margin=1in]{geometry}
\usepackage[utf8]{inputenc}
\usepackage{amsmath}
\usepackage{amsthm}
\usepackage{amssymb}
\usepackage{amsfonts}
\usepackage[colorlinks,allcolors=blue]{hyperref}
\usepackage{microtype}
\usepackage{graphicx}

%\addtokomafont{disposition}{\rmfamily}
\let\emptyset=\varnothing

\newcommand{\tb}{\textbf}
\newcommand{\tbu}[1]{\tb{\textup{#1}}}

\newcommand{\dpar}[1]{\left(#1\right)}
\newcommand{\dsqr}[1]{\left[#1\right]}
\newcommand{\dcur}[1]{\left\{#1\right\}}
\newcommand{\dabs}[1]{\left|#1\right|}
\newcommand{\ang}[1]{\left\langle#1\right\rangle}

\newcommand{\ds}{\displaystyle}

\newcommand{\N}{\mathbb{N}}
\newcommand{\Z}{\mathbb{Z}}
\newcommand{\Q}{\mathbb{Q}}
\newcommand{\R}{\mathbb{R}}

\DeclareMathOperator{\sen}{sen}
\DeclareMathOperator{\arcsen}{arc sen}
\DeclareMathOperator{\senh}{senh}

\title{Primeira lista de matemática}
\author{Prof.: Max Jáuregui}
\date{}

%\linespread{1.2}
%\linespread{1.213} %11pt
%\linespread{1.241} %12pt

\begin{document}
\maketitle
\begin{enumerate}
  \item Escreva explicitamente os elementos dos seguintes conjuntos:
  \begin{enumerate}
    \item $A=\{n\in\N:n\le 5\}$
    \item $B=\{n\in\N:n<5\}$
    \item $C=\{n\in\Z:-2<n\le 2\}$
    \item $D=\{m^2\in\R:m\in\Z, -1\le m\le 2\}$
    \item $E=\{m/n\in\Q:m\in\{1,2,3\},n\in\{1,2\}\}$
  \end{enumerate}
  \item Dados os conjuntos $X=\{2,3,5,7\}$ e $Y=\{n\in\Z:-1\le x\le 3\}$, encontre $X\cup Y$ (união), $X\cap Y$ (interseção), $X\setminus Y$ (diferença de $X$ com $Y$) e $Y\setminus X$ (diferença de $Y$ com $X$).
  \item Use diagramas para verificar o seguinte:
  \begin{enumerate}
    \item $A\cap B\subset B$
    \item $A\setminus B\subset A$
    \item $A\cup(B\cap A)=A$
    \item $A\setminus(A\cap B)=A\setminus B$
    \item $A\cup(B\cap C)=(A\cup B)\cap (A\cup C)$
  \end{enumerate}
  \item Quantos números de $3$ dígitos podem ser formados com os algarismos $2,3$ e $5$?
  \item No exercício anterior, se queremos números cujos algarismos sejam todos diferentes, quantos números haveriam?
  \item De quantas formas podem ser colocados $5$ relógios distintos em duas caixas, sabendo que cada caixa pode conter mais de um relógio?
  \item Suponha que em um estacionamento haja $5$ vagas disponíveis. De quantas formas diferentes podem ser colocados $3$ carros?
  \item Encontre o número total de ordenamentos possíveis das letras das palavras (a)~ROMA, (b) CASA e (c) GARRAFA.
\end{enumerate}
\end{document}

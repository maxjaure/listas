\documentclass[12pt,a4paper]{article}
\usepackage[portuguese]{babel}
\usepackage[margin=1in]{geometry}
\usepackage[utf8]{inputenc}
\usepackage{amsmath}
\usepackage{amsthm}
\usepackage{amssymb}
\usepackage{amsfonts}
\usepackage[colorlinks,allcolors=blue]{hyperref}
\usepackage{microtype}
\usepackage{graphicx}

\let\emptyset=\varnothing

\newcommand{\tb}{\textbf}
\newcommand{\tbu}[1]{\tb{\textup{#1}}}
\newcommand{\mb}{\mathbf}
\newcommand{\mc}{\mathcal}

\newcommand{\dpar}[1]{\left(#1\right)}
\newcommand{\dsqr}[1]{\left[#1\right]}
\newcommand{\dcur}[1]{\left\{#1\right\}}
\newcommand{\dabs}[1]{\left|#1\right|}
\newcommand{\ang}[1]{\left\langle#1\right\rangle}

\newcommand{\ds}{\displaystyle}

\newcommand{\N}{\mathbb{N}}
\newcommand{\Z}{\mathbb{Z}}
\newcommand{\Q}{\mathbb{Q}}
\newcommand{\R}{\mathbb{R}}

\DeclareMathOperator{\sen}{sen}
\DeclareMathOperator{\arcsen}{arc sen}
\DeclareMathOperator{\senh}{senh}
\DeclareMathOperator{\tr}{tr}
\DeclareMathOperator{\posto}{posto}
\DeclareMathOperator{\sgn}{sgn}

\title{Primeira lista de álgebra linear}
\author{Prof.: Max Jáuregui}
\date{}
%\linespread{1.2}
%\linespread{1.213} %11pt
%\linespread{1.241} %12pt

\begin{document}
\maketitle
\begin{enumerate}
  \item Construa uma matriz $4\times4$ antissimétrica e não-nula. Quantos elementos da matriz você tem liberdade de escolher?
  \item Construa uma matriz $2\times 2$ hermitiana cujos elementos não sejam todos reais e cujo traço seja $1$.
  \item Construa uma matriz triangular inferior $3\times 3$ e uma matriz simétrica $4\times 4$, ambas não-nulas e que tenham o mesmo traço.
  \item Usando o símbolo de Kronecker, escreva os elementos de uma matriz diagonal arbitrária de ordem $n\times n$.
  \item Dadas as matrizes
  $$\mb a=\begin{bmatrix}
    2&0&3\\
    1&4&-1
  \end{bmatrix}\,,\quad \mb b=\begin{bmatrix}
    -2&3&0\\
    1&3&2
  \end{bmatrix}\quad\text{e}\quad \mb c=\begin{bmatrix}
    0&1&0\\
    -1&0&-1
  \end{bmatrix}\,,$$
  calcule (i) $\mb a+2\mb b$, (ii) $2\mb a^T-\mb b^T$, (iii) $2\mb b+2\mb c$, (iv) $\mb a^T\mb b$, (v) $\mb b\mb c^T$, (vi) $\mb a^T\mb c+\mb b^T\mb c$.
  \item Seja $\mb a$ uma matriz $2\times 3$. Se
  $$\mb b=\begin{bmatrix}
    1&0&a_{11}&a_{12}&a_{13}\\
    0&1&a_{21}&a_{22}&a_{23}\\
    0&0&1&0&0\\
    0&0&0&1&0\\
    0&0&0&0&1
  \end{bmatrix}=\begin{bmatrix}
    \mb I_2&\mb a\\
    \mb 0&\mb I_3
  \end{bmatrix}\,,$$
  mostre que
  $$\mb b^2=\begin{bmatrix}
    \mb I_2&2\mb a\\
    \mb 0&\mb I_3
  \end{bmatrix}\,.$$
  Isso ilustra que é possível multiplicar matrizes por blocos.
  \item Dadas as matrizes
  $$\mb a=\begin{bmatrix}
    5&0\\
    2&1
  \end{bmatrix}\,,\quad \mb b=\begin{bmatrix}
    1&0\\
    -2&5
  \end{bmatrix}\quad\text{e}\quad \mb c=\begin{bmatrix}
    2&-3&6&-7\\
    1&-4&5&-8
  \end{bmatrix}\,,$$
  calcule $\mb a(\mb{bc})$.
  \item Mostre que $\sum_{j=1}^na_i\delta_{ij}b_j\delta_{jk}=a_ib_i\delta_{ik}$, em que os índices $i$ e $k$ são fixos. Usando isso, conclua que o produto de duas matrizes diagonais $\mb a$ e $\mb b$ da mesma ordem é uma matriz diagonal cujos elementos da diagonal são obtidos multiplicando os elementos da diagonal de $\mb a$ com os de $\mb b$.
  \item Use o método de eliminação para transformar a seguinte matriz em uma matriz escalonada:
  $$\begin{bmatrix}
    0&2&-1&1\\
    1&0&2&3\\
    2&-1&-1&0
  \end{bmatrix}$$
  \item Determine o posto da seguinte matriz:
  $$\begin{bmatrix}
    -1&4&2\\
    2&0&3\\
    0&8&7\\
    3&-4&1
  \end{bmatrix}$$
  \item Escreva as matrizes elementares correspondentes às operações elementares (i) $L_3+2L_1$, (ii) $L_2\leftrightarrow L_3$, (iii) $L_1/5$, aplicadas a uma matriz $3\times 4$.
  \item Mostre que o seguinte sistema linear não tem solução:
  \begin{equation*}
    \begin{split}
      3x-2y+z&=2\\
      2x+4y-3z&=3\\
      5x-14y+9z&=1\,.
    \end{split}
  \end{equation*}
  \item Mostre que o seguinte sistema linear tem como única solução a lista $(1,0,0,0)$:
  \begin{equation*}
    \begin{split}
      2x-y&=2\\
      y+z-t&=0\\
      x+z-t&=1\\
      2y+z+t&=0\,.
    \end{split}
  \end{equation*}
  \item Determine a inversa da matriz
  $$\mb a=\begin{bmatrix}
    -2&0&1\\
    4&1&3\\
    1&0&-3
  \end{bmatrix}\,.$$
\end{enumerate}
\end{document}

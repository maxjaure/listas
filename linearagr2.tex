\documentclass[12pt,a4paper]{article}
\usepackage[portuguese]{babel}
\usepackage[margin=1in]{geometry}
\usepackage[utf8]{inputenc}
\usepackage{amsmath}
\usepackage{amsthm}
\usepackage{amssymb}
\usepackage{amsfonts}
\usepackage[colorlinks,allcolors=blue]{hyperref}
\usepackage{microtype}
\usepackage{graphicx}

\let\emptyset=\varnothing

\newcommand{\tb}{\textbf}
\newcommand{\tbu}[1]{\tb{\textup{#1}}}
\newcommand{\mb}{\mathbf}
\newcommand{\mc}{\mathcal}

\newcommand{\dpar}[1]{\left(#1\right)}
\newcommand{\dsqr}[1]{\left[#1\right]}
\newcommand{\dcur}[1]{\left\{#1\right\}}
\newcommand{\dabs}[1]{\left|#1\right|}
\newcommand{\ang}[1]{\left\langle#1\right\rangle}

\newcommand{\ds}{\displaystyle}

\newcommand{\N}{\mathbb{N}}
\newcommand{\Z}{\mathbb{Z}}
\newcommand{\Q}{\mathbb{Q}}
\newcommand{\R}{\mathbb{R}}

\DeclareMathOperator{\sen}{sen}
\DeclareMathOperator{\arcsen}{arc sen}
\DeclareMathOperator{\senh}{senh}
\DeclareMathOperator{\tr}{tr}
\DeclareMathOperator{\posto}{posto}
\DeclareMathOperator{\sgn}{sgn}

\title{Segunda lista de álgebra linear}
\author{Prof.: Max Jáuregui}
\date{}
%\linespread{1.2}
%\linespread{1.213} %11pt
%\linespread{1.241} %12pt

\begin{document}
\maketitle
\begin{enumerate}
\item Determine o sinal das seguintes permutações da lista $(1,2,3,4,5,6)$:
\begin{enumerate}
\item $(6,5,4,3,2,1)$
\item $(4,5,2,1,3,6)$
\item $(2,1,4,3,6,5)$
\end{enumerate}
\item Justifique por que o determinante de uma matriz quadrada qualquer que tem uma coluna nula é igual a zero.
\item Calcule os seguintes determinantes:
\begin{enumerate}
\item $\begin{vmatrix}1&3\\4&5\end{vmatrix}$
\item $\begin{vmatrix}1&3&2\\1&5&5\\0&2&3\end{vmatrix}$
\item $\begin{vmatrix}3&0&4&2\\5&1&2&0\\0&-1&4&0\\0&0&2&0\end{vmatrix}$
\end{enumerate}
\item Mostre que 
$$p(x)=\begin{vmatrix}1&1&1\\2&3&x\\4&9&x^2\end{vmatrix}$$ é um polinômio cujas raízes são $2$ e $3$.
\item Dê um exemplo de uma matriz $2\times 2$ que não seja triangular e que seja invertível.
\item Dê um exemplo de uma matriz $3\times 3$ que não seja triangular e que seja invertível.
\item Dadas as matrizes
$$\mb a=\begin{bmatrix} 3&2&1\\4&3&2\\1&-1&-3\end{bmatrix}\quad\text{e}\quad\mb b=\begin{bmatrix} 1&9&3\\2&5&7\\-5&4&8\end{bmatrix}\,,$$
calcule $\det(\mb{ab})$ e $\det(\mb{ba})$.
\item Use a regra de Cramer para resolver o seguinte sistema linear:
\begin{equation*}
\begin{split}
5x-3y&=10\\
4x+7y&=2\,.
\end{split}
\end{equation*}
\item Usando determinantes calcule a inversa das seguintes matrizes:
\begin{enumerate}
\item $\begin{bmatrix} 5&2\\7&1\end{bmatrix}$
\item $\begin{bmatrix} 3&2&1\\3&4&5\\1&-1&2\end{bmatrix}$
\end{enumerate}
\end{enumerate}
\end{document}

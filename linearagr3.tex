\documentclass[a4paper]{article}
\usepackage[portuguese]{babel}
\usepackage[utf8]{inputenc}
\usepackage{amsmath}
\usepackage{amsfonts}
\usepackage[colorlinks,allcolors=blue]{hyperref}
\usepackage{microtype}
\usepackage{graphicx}

\let\emptyset=\varnothing

\newcommand{\tb}{\textbf}
\newcommand{\tbu}[1]{\tb{\textup{#1}}}
\newcommand{\mb}{\mathbf}
\newcommand{\mc}{\mathcal}

\newcommand{\dpar}[1]{\left(#1\right)}
\newcommand{\dsqr}[1]{\left[#1\right]}
\newcommand{\dcur}[1]{\left\{#1\right\}}
\newcommand{\dabs}[1]{\left|#1\right|}
\newcommand{\ang}[1]{\left\langle#1\right\rangle}

\newcommand{\ds}{\displaystyle}

\newcommand{\N}{\mathbb{N}}
\newcommand{\Z}{\mathbb{Z}}
\newcommand{\Q}{\mathbb{Q}}
\newcommand{\R}{\mathbb{R}}

\DeclareMathOperator{\sen}{sen}
\DeclareMathOperator{\arcsen}{arc sen}
\DeclareMathOperator{\senh}{senh}
\DeclareMathOperator{\tr}{tr}
\DeclareMathOperator{\posto}{posto}
\DeclareMathOperator{\sgn}{sgn}

\title{Terceira lista de matemática II}
\author{Prof.: Max Jáuregui}
\date{}
%\linespread{1.2}
%\linespread{1.213} %11pt
%\linespread{1.241} %12pt

\begin{document}
\maketitle
\begin{enumerate}
\item Sejam os vetores bidimensionais $\vec A=(1,3)$, $\vec B=(2,5)$ e $\vec C=(3,-1)$. Faça o seguinte:
\begin{enumerate}
	\item Represente graficamente o vetor $\vec A$ colocando sua origem na origem do sistema de coordenadas.
	\item Represente graficamente o vetor $\vec B$ colocando sua origem no ponto $(3,-1)$.
	\item Usando as coordenadas calcule a soma $\vec A+\vec B+\vec C$. Por outro lado, construa o vetor soma de forma gráfica e compare seus resultados.
	\item Determine o vetor $2\vec A-3\vec B+\vec C$.
	\item Encontre um vetor unitário paralelo a $\vec A$.
\end{enumerate}
\item Mostre que o vetor tridimensional $\vec A=(1,3,1)$ pode ser escrito como uma combinação linear dos vetores $\vec B=(1,1,1)$ e $\vec C=(2,3,2)$.
\item Dê um exemplo de um conjunto $X\subset\R^2$ que tenha dois vetores L.I. cujas primeiras coordenadas sejam iguais a $1$.
\item Mostre que o conjunto $X=\{(1,2,3),(4,5,6),(7,6,8)\}\subset\R^3$ é L.D..
\item Mostre que o conjunto $\mc B=\{(-1,1,2),(1,1,1),(1,-1,-1)\}\subset\R^3$ é uma base de $\R^3$.
\item Escreva o vetor $(1,0,0)\in\R^3$ como uma combinação linear dos vetores da base $\mc B$ do exercício anterior.
\item Dados os vetores tridimensionais $\vec A=(1,2,3)$, $\vec B=(-1,0,1)$ e $\vec C=(1,2,1)$. Faça o seguinte:
\begin{enumerate}
	\item Calcule $\vec A\cdot\vec B$ e $\vec A\cdot \vec C$.
	\item Mostre que os vetores $\vec B$ e $\vec C$ são ortogonais.
	\item Calcule $|\vec B+\vec C|^2$. Compare seu resultado com $|\vec B|^2+|\vec C|^2$.
	\item Determine o ângulo entre os vetores $\vec A$ e $\vec B$.
	\item Calcule $|\vec A-\vec B|^2$. Compare seu resultado com $|\vec A|^2+|\vec B|^2-2\vec A\cdot\vec B$.
\end{enumerate} 
\item Considere um triângulo no qual dois dos seus lados têm comprimentos $3$ e $5$ respectivamente e formam um ângulo de $60º$. Com essas informações, determine o comprimento do terceiro lado.
\item Dados os vetores tridimensionais $\vec A=(2,1,1)$, $\vec B=(1,1,1)$ e $\vec C=(2,2,2)$. Faça o seguinte:
\begin{enumerate}
	\item Calcule $\vec A\times \vec B$, $\vec A\times \vec C$ e $\vec B\times \vec C$.
	\item Calcule $\vec B\times \vec A$, $\vec C\times \vec A$ e $\vec C\times \vec B$. Compare com o item anterior.
	\item Calcule $\vec A\cdot (\vec A\times\vec B)$ e $\vec B\cdot(\vec A\times\vec B)$.
	\item Determine a área do paralelogramo construído a partir dos vetores $\vec A$ e $\vec C$.
\end{enumerate}
\item Dê um exemplo de um vetor não-nulo que seja ortogonal ao vetor tridimensional $\vec A=(1,2,3)$.
\item Dê um exemplo de um vetor não-nulo que seja ortogonal aos vetores tridimensionais $\vec A=(1,1,1)$ e $\vec B=(1,0,1)$ simultaneamente.
\item Encontre o volume do paralelepípedo construído a partir dos vetores tridimensionais $\vec A=(1,2,1)$, $\vec B=(-1,2,1)$ e $\vec C=(0,-1,2)$.
\end{enumerate}
\end{document}

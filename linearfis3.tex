\documentclass[10pt,a4paper]{article}
\usepackage[portuguese]{babel}
%\usepackage[margin=1in]{geometry}
\usepackage{amsmath}
\usepackage[utf8]{inputenc}
\usepackage{amsthm}
\usepackage{amsfonts}
\usepackage[colorlinks,allcolors=blue]{hyperref}
\usepackage{microtype}
\usepackage{graphicx}

\let\emptyset=\varnothing

\newcommand{\tb}{\textbf}
\newcommand{\tbu}[1]{\tb{\textup{#1}}}
\newcommand{\mb}{\mathbf}
\newcommand{\mc}{\mathcal}

\newcommand{\dpar}[1]{\left(#1\right)}
\newcommand{\dsqr}[1]{\left[#1\right]}
\newcommand{\dcur}[1]{\left\{#1\right\}}
\newcommand{\dabs}[1]{\left|#1\right|}
\newcommand{\ang}[1]{\left\langle#1\right\rangle}

\newcommand{\ds}{\displaystyle}

\newcommand{\N}{\mathbb{N}}
\newcommand{\Z}{\mathbb{Z}}
\newcommand{\Q}{\mathbb{Q}}
\newcommand{\R}{\mathbb{R}}

\DeclareMathOperator{\sen}{sen}
\DeclareMathOperator{\arcsen}{arc sen}
\DeclareMathOperator{\senh}{senh}
\DeclareMathOperator{\tr}{tr}
\DeclareMathOperator{\posto}{posto}
\DeclareMathOperator{\sgn}{sgn}

\title{Terceira lista de álgebra linear}
\author{Prof.: Max Jáuregui}
\date{}
%\linespread{1.2}
%\linespread{1.213} %11pt
%\linespread{1.241} %12pt

\begin{document}
\maketitle
\begin{enumerate}
\item Considere o conjunto $\R^2$ com a definição usual de multiplicação por um número real. Mostre que $\R^2$ não é um espaço vetorial se a adição é definida por
\begin{enumerate}
	\item $u+v=(x_1+x_2,0)$
	\item $u+v=(x_1x_2,y_1y_2)$
\end{enumerate}
para quaisquer $u=(x_1,y_1)$ e $v=(x_2,y_2)$.
\item Seja $E$ um espaço vetorial. Usando as propriedades de espaço vetorial e o fato de que $u+v=u+w\Rightarrow v=w$ para quaisquer $u,v,w\in E$, mostre que $(-1)v=-v$.
\item Sejam $u=(a,b)$ e $v=(c,d)$ vetores não-nulos de $\R^2$. Mostre que $u$ é um múltiplo de $v$ ($u$ é paralelo a $v$) se, e somente se, $ad-bc=0$.
\item Mostre que o conjunto $S$ das matrizes simétricas $3\times 3$ é um subespaço do espaço vetorial $M(3\times 3)$ das matrizes $3\times 3$.
\item Mostre que o conjunto $P$ das funções pares $f:\R\to\R$ é um subespaço de $\R^\R$.
\item Encontre o subespaço gerado pelas funções polinomiais $p_0,p_1,\ldots,p_n:\R\to\R$ definidas por $p_k(x)=x^k$, $k\in\{0,1,\ldots,n\}$.
\item Mostre que o conjunto $H=\{(x_1,\ldots,x_n)\in\R^n:a_1x_1+\cdots+a_nx_n=0\}$ é um subespaço de $\R^n$. Esse subespaço é chamado de um \emph{hiperplano} de $\R^n$.
\item Mostre que o sistema de equações lineares
\begin{equation*}
	\begin{split}
	a_{11}x_1+\cdots+a_{1n}x_n&=0\\
	\vdots\qquad\qquad&\quad\;\;\vdots\\
	a_{m1}x_1+\cdots+a_{mn}x_n&=0
	\end{split}
\end{equation*}
sempre tem solução (encontre uma solução particular que não depende dos valores dos coeficientes $a_{ij}$). Além disso, mostre que o conjunto das soluções desse sistema é um subespaço de $\R^n$ (perceba que cada equação do sistema está associada a um hiperplano de $\R^n$).
\item É possível escrever o vetor $(2,3,6)$ como combinação linear dos vetores $(1,2,1)$ e $(0,4,1)$?
\item Usando o exercício 3 e a regra de Cramer, mostre que se $u,v\in\R^2$ são vetores não-nulos e não-paralelos, então o subespaço gerado por $\{u,v\}$ é $\R^2$, ou seja, $\{u,v\}$ é um conjunto de geradores de $\R^2$.
\item Encontre dois conjuntos de geradores diferentes para o espaço vetorial $E=\{(\alpha,-\alpha)\in\R^2:\alpha\in\R\}$. É possível achar um conjunto de geradores de $E$ que contenha mais de um vetor?
\end{enumerate}
\end{document}

\documentclass[a4paper]{article}
\usepackage[portuguese]{babel}
\usepackage[utf8]{inputenc}
\usepackage{amsmath}
\usepackage{amsfonts}
\usepackage[colorlinks,allcolors=blue]{hyperref}
\usepackage{microtype}
\usepackage{graphicx}

\let\emptyset=\varnothing

\newcommand{\tb}{\textbf}
\newcommand{\tbu}[1]{\tb{\textup{#1}}}
\newcommand{\mb}{\mathbf}
\newcommand{\mc}{\mathcal}

\newcommand{\dpar}[1]{\left(#1\right)}
\newcommand{\dsqr}[1]{\left[#1\right]}
\newcommand{\dcur}[1]{\left\{#1\right\}}
\newcommand{\dabs}[1]{\left|#1\right|}
\newcommand{\ang}[1]{\left\langle#1\right\rangle}

\newcommand{\ds}{\displaystyle}

\newcommand{\N}{\mathbb{N}}
\newcommand{\Z}{\mathbb{Z}}
\newcommand{\Q}{\mathbb{Q}}
\newcommand{\R}{\mathbb{R}}

\DeclareMathOperator{\sen}{sen}
\DeclareMathOperator{\arcsen}{arc sen}
\DeclareMathOperator{\senh}{senh}
\DeclareMathOperator{\tr}{tr}
\DeclareMathOperator{\posto}{posto}
\DeclareMathOperator{\sgn}{sgn}

\title{Quarta lista de álgebra linear}
\author{Prof.: Max Jáuregui}
\date{}
%\linespread{1.2}
%\linespread{1.213} %11pt
%\linespread{1.241} %12pt

\begin{document}
\maketitle
\begin{enumerate}
\item Dê um exemplo de um conjunto $X\subset\R^2$ que tenha dois vetores L.I. cujas primeiras coordenadas sejam iguais a $1$.
\item Mostre que o conjunto $X=\{(1,2,3),(4,5,6),(7,6,8)\}\subset\R^3$ é L.D..
\item Mostre que o conjunto $\mc B=\{(-1,1,2),(1,1,1),(1,-1,-1)\}\subset\R^3$ é uma base de $\R^3$.
\item Escreva o vetor $(1,0,0)\in\R^3$ como uma combinação linear dos vetores da base $\mc B$ do exercício anterior.
\item Seja $M(2\times 2)$ o espaço vetorial das matrizes $2\times 2$. Qual é a dimensão desse espaço? Adicionalmente, mostre que o conjunto
  $$\mc B=\dcur{
    \begin{bmatrix}
      1&0\\
      0&1
    \end{bmatrix},\begin{bmatrix}
      0&1\\
      1&0
    \end{bmatrix},\begin{bmatrix}
      0&-1\\
      1&0
    \end{bmatrix},\begin{bmatrix}
      1&0\\
      0&-1
    \end{bmatrix}
  }$$
  é uma base de $M(2\times 2)$.
\item Escreva a matriz $
  \begin{bmatrix}
    1&0\\
    0&0
  \end{bmatrix}
$ como uma combinação linear das matrizes da base $\mc B$ do exercício anterior.
\item Mostre que o conjunto $X=\{(1,0,1,0),(1,2,3,4),(0,0,1,1)\}\subset\R^4$ é L.I.. Quál é a dimensão do subespaço gerado por $X$? É menor do que $4$?
\item Seja $\mc P_3$ o espaço vetorial dos polinômios de grau $\le 3$. Quál é a dimensão desse espaço? Adicionalmente, mostre que $\mc B=\{x^2-2x+1,x^3,x^3-x,4\}$ é uma base de $\mc P_3$.
\item Seja $\R^\R$ o espaço vetorial das funções reais de uma variável real. Mostre que os seguintes subconjuntos de $\R^\R$ são L.I.
  \begin{enumerate}
  \item $\{\sen x,\cos x\}$
  \item $\{e^x,e^{2x},e^{3x}\}$
  \item $\{\sen x,\sen 2x,\sen 3x\}$
  \end{enumerate}
  \textit{Dica:} Use derivada.
\item O espaço $\R^\R$ tem dimensão infinita. Quais são as dimensões dos subespaços gerados pelos subconjuntos de $\R^\R$ considerados no exercício anterior.
\end{enumerate}
\end{document}

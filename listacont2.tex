\documentclass[12pt,a4paper]{article}
\usepackage[portuguese]{babel}
\usepackage[margin=1in]{geometry}
\usepackage[utf8]{inputenc}
\usepackage{amsmath}
\usepackage{amsthm}
\usepackage{amssymb}
\usepackage{amsfonts}
\usepackage[colorlinks,allcolors=blue]{hyperref}
\usepackage{microtype}
\usepackage{graphicx}

%\addtokomafont{disposition}{\rmfamily}
\let\emptyset=\varnothing

\newcommand{\tb}{\textbf}
\newcommand{\tbu}[1]{\tb{\textup{#1}}}

\newcommand{\dpar}[1]{\left(#1\right)}
\newcommand{\dsqr}[1]{\left[#1\right]}
\newcommand{\dcur}[1]{\left\{#1\right\}}
\newcommand{\dabs}[1]{\left|#1\right|}
\newcommand{\ang}[1]{\left\langle#1\right\rangle}

\newcommand{\ds}{\displaystyle}

\newcommand{\N}{\mathbb{N}}
\newcommand{\Z}{\mathbb{Z}}
\newcommand{\Q}{\mathbb{Q}}
\newcommand{\R}{\mathbb{R}}

\DeclareMathOperator{\sen}{sen}
\DeclareMathOperator{\arcsen}{arc sen}
\DeclareMathOperator{\senh}{senh}

\title{Segunda lista de matemática}
\author{Prof.: Max Jáuregui}
\date{}

%\linespread{1.2}
%\linespread{1.213} %11pt
%\linespread{1.241} %12pt

\begin{document}
\maketitle
\begin{enumerate}
  \item Calcule os seguintes coeficientes binomiais: (a) $\binom{5}{3}$, (b) $\binom{7}{4}$, (c) $\binom{7}{3}$.
  \item No truco, $3$ cartas são dadas a um jogador de um baralho contendo $40$ cartas. Determine o número de possíveis situações diferentes que o jogador pode encontrar. (Resposta: $9880$)
  \item De quantas formas distintas podem ser colocados $4$ carros indistinguíveis em um estacionamento que tem $8$ vagas? (Resposta: $70$)
  \item Encontre o número de possíveis resultados diferentes no lançamento simultâneo de $5$ moedas idênticas assumindo que cada moeda pode sair ou cara ou coroa. (Resposta:~$6$)
  \item Expanda o binômio $(x-3)^4$.
  \item Expanda o polinômio $(2x+1)^3+(x-2)^2$.
  \item Encontre o coeficiente do termo em $x^6$ na expansão do binômio $(x-2)^9$.
  \item Determine o coeficiente do termo em $x^6$ na expansão do binômio $(x^2-1)^5$.
  \item Encontre o coeficiente do termo em $x^4$ na expansão do polinômio $(x^2-2)^4-(x+3)^5$.
\end{enumerate}
\end{document}

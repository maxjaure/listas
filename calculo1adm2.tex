\documentclass[12pt,a4paper]{article}
\usepackage[portuguese]{babel}
\usepackage[margin=1in]{geometry}
\usepackage[utf8]{inputenc}
\usepackage{amsmath}
\usepackage{amsthm}
\usepackage{amssymb}
\usepackage{amsfonts}
\usepackage[colorlinks,allcolors=blue]{hyperref}
\usepackage{microtype}
\usepackage{graphicx}

%\addtokomafont{disposition}{\rmfamily}
\let\emptyset=\varnothing

\newcommand{\tb}{\textbf}
\newcommand{\tbu}[1]{\tb{\textup{#1}}}

\newcommand{\dpar}[1]{\left(#1\right)}
\newcommand{\dsqr}[1]{\left[#1\right]}
\newcommand{\dcur}[1]{\left\{#1\right\}}
\newcommand{\dabs}[1]{\left|#1\right|}
\newcommand{\ang}[1]{\left\langle#1\right\rangle}

\newcommand{\ds}{\displaystyle}

\newcommand{\N}{\mathbb{N}}
\newcommand{\Z}{\mathbb{Z}}
\newcommand{\Q}{\mathbb{Q}}
\newcommand{\R}{\mathbb{R}}

\DeclareMathOperator{\sen}{sen}
\DeclareMathOperator{\arcsen}{arc sen}
\DeclareMathOperator{\senh}{senh}

\title{Segunda lista de matemática II}
\author{Prof.: Max Jáuregui}
\date{}

%\linespread{1.2}
%\linespread{1.213} %11pt
%\linespread{1.241} %12pt

\begin{document}
\maketitle

\begin{enumerate}
  \item Dadas as funções $f,g:\R\to\R$ definidas por $f(x)=x^2-2x$ e $g(x)=2x+3$. Encontre as expressões das funções compostas $f\circ g$ e $g\circ f$.
  \item Dada a função $f:[1,\infty)\to[0,\infty)$ definida por $f(x)=\sqrt{3(x-1)}$, verifique que a função $g:[0,\infty)\to[1,\infty)$ definida por $\ds g(x)=\frac{x^2}{3}+1$ é a inversa de $f$.
  \item Encontre as inversas das seguintes funções:
  \begin{enumerate}
    \item $f:\R\to\R$, $f(x)=2x+7$
    \item $g:[-1,\infty)\to [0,\infty)$, $g(x)=\sqrt{x+1}$
    \item $h:[0,\infty)\to (-\infty,2]$, $h(x)=2-x^2$
  \end{enumerate}
  \item Determine o domínio das funções: $f(x)=2x-4$, $g(x)=3$, $h(x)=3x^2+7x-2$, $p(x)=(x-2)^5+3x^3+2x$.
  \item Construa o gráfico das seguintes funções afins:
  \begin{enumerate}
    \item $f(x)=2x-3$
    \item $g(x)=-3x+1$
    \item $\ds h(x)=\frac{x}{2}+\frac{3}{4}$
    \item $\ds r(x)=2-\frac{3x}{2}$
  \end{enumerate}
  \item Determine a imagem das seguintes funções quadráticas:
  \begin{enumerate}
    \item $f(x)=x^2-x+1$
    \item $g(x)=3x^2+2x+2$
    \item $h(x)=-2x^2+4x-1$
    \item $r(x)=-3x^2-3x+2$
  \end{enumerate}
  Além disso, esboce o gráfico de cada uma dessas funções, indicando a posição do seu ponto extremo (máximo ou mínimo).
  \item Determine o domínio das seguintes funções:
  \begin{enumerate}
    \item $\ds f(x)=\frac{3x^2+2x-5}{3x-7}$
    \item $\ds g(x)=\frac{x^3-2x+1}{4(2x-1)(x+5)}$
    \item $\ds h(x)=\frac{x^4-3x^2+6}{x^2-4x+3}$
    \item $\ds p(x)=\sqrt{5x+8}$
    \item $\ds q(x)=\sqrt{(2-3x)(x+1)}$
    \item $\ds r(x)=\sqrt{x^2+2x-3}$
    \item $\ds s(x)=\sqrt{2+3x-x^2}$
  \end{enumerate}
  \item Esboce o gráfico das seguintes funções:
  \begin{enumerate}
    \item $f(x)=\sqrt{2x+3}$
    \item $\ds g(x)=\frac{1}{2x-1}$
  \end{enumerate}
  \item Esboce o gráfico da função $f(x)=x^3-x+1$ e a partir daí conclua que a equação $x^3-x+1=0$ tem uma única solução.
\end{enumerate}
\end{document}
